\documentclass[sigconf]{acmart}

\usepackage{booktabs} % For formal tables
\newcommand{\red}[1]{\textcolor{red}{#1}}
\newcommand{\commentary}[1]{\noindent \red{{\bf Commentary}: #1}}

\setcopyright{none}
\settopmatter{printacmref=false} % Removes citation information below abstract
\renewcommand\footnotetextcopyrightpermission[1]{} % removes footnote with conference information in first column
\pagestyle{plain} % removes running headers

\begin{document}

\title{The Anserini Docker Image for OSIRRC at SIGIR 2019}

\author{Ryan Clancy and Jimmy Lin}
\affiliation{\vspace{0.1cm}
  \department{David R. Cheriton School of Computer Science}
  \institution{University of Waterloo}
}

%  \begin{abstract}
%    Abstract goes here.
%  \end{abstract}

  %\keywords{datasets, neural networks, gaze detection, text tagging}

\maketitle

%\section{Overview}

\noindent {\bf Image Source:} \texttt{\small https://github.com/osirrc/anserini-docker} \\
{\bf Docker Hub:} \texttt{\small https://hub.docker.com/r/osirrc2019/anserini} \\

\commentary{My suggestion for these ``docker papers'' is to provide the URLs to the implementations and leave technical details there (i.e., don't repeat stuff that's better captured in the README of the source image repo.}

\section{Overview}

Anserini~\cite{Yang_etal_SIGIR2017,Yang_etal_JDIQ2018} is an open-source information retrieval toolkit built around Lucene to facilitate replicable research.
The project grew out of the Open-Source IR Reproducibility Challenge from 2015~\cite{Lin_etal_ECIR2016} and reflects growing community interest in using Lucene for academic IR research~\cite{azzopardi2016lucene4ir,Azzopardi_etal_SIGIR2017}.
As Lucene was not originally designed as a research toolkit, Anserini aims to fill in ``the missing parts'' that allow researchers to run standard {\it ad hoc} retrieval experiments ``right out of the box'', including competitive baselines and integration hooks for neural ranking models.
Given Lucene's tremendous production deployment base (typically via Solr or Elasticsearch), better alignment between research in information retrieval and the practice of building research world search engines promises a smoother transition path from the lab to the ``real world'' for research innovations.

\smallskip
\commentary{Here, my suggestion is to provide an overview of the underlying system, techniques, etc.}

\section{Technical Design}

\noindent {\bf Supported Collections:}\\
\texttt{robust4}, \texttt{core17}, \texttt{core18},
\texttt{gov2}, \texttt{cw09b}, \texttt{cw12b}

\smallskip \noindent {\bf Supported Hooks:}\\
\texttt{init}, \texttt{index}, \texttt{search}

\smallskip \noindent
As with Lucene itself, Anserini is implemented in Java.
Thus, the Anserini Docker image primarily consisted of lightweight hooks implemented in Python that wraps underlying Java commands.

\smallskip
\commentary{Here, my suggestion is to provide an overview of the technical design of the image, without going into too much detail.}

\section{Retrieval Models}

The Anserini Docker image 

\smallskip
\commentary{Here, my suggestion is to provide an overview of the retrieval models...}


\section{OSIRRC Experience}

Anserini was the first image that was created in conjunction with the jig...

\smallskip
\commentary{Like, how was the experience?}


\bibliographystyle{ACM-Reference-Format}
\bibliography{anserini-docker}

\end{document}

\endinput
